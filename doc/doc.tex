\documentclass{article}

\usepackage{mystyle}

\title{Univents}
\author{Davide P. -- Deborah A. -- Elia T.}
\date{}

\begin{document}
%
\maketitle
%
\section*{Contesto}
%
In una città universitaria come Trento sono numerosi gli eventi indirizzati agli studenti -- incontri culturali come seminari e conferenze, o più informali come un aperitivo in compagnia -- e può risultare difficile tenere traccia di tutte le iniziative che hanno luogo in questo ampio contesto, rischiando che le occasioni interessanti passino silenziosamente inosservate.

Per far fronte alla crescente complessità organizzativa e gestionale della comunità studentesca proponiamo \textsc{Univents}, una piattaforma online che raccoglie gli eventi e le iniziative offerte sia dall'ateneo che dagli studenti, proponendole in un'interfaccia semplice e intuitiva che incoraggi la comunicazione e la condivisione di contenuti.
%
\section*{Obiettivi}
%
La comunità studentesca di Trento è molto numerosa ed è estremamente importante che sia un gruppo unito e coeso, per garantire un'esperienza universitaria sana e stimolante.

Per ottenere tale scopo, \textsc{Univents} fornisce una piattaforma dove poter consultare facilmente e rapidamente la lista degli eventi che si tengono nella città di Trento, promuovendo la partecipazione a tali eventi e di conseguenza lo sviluppo della vita sociale universitaria all'interno della città. In particolare, gli obiettivi sono:
%
\begin{enumerate}
   \item Promuovere la vita sociale universitaria all'interno della città di Trento e la partecipazione agli eventi che vengono tenuti
   \item Elencare gli eventi in programma in un'interfaccia facile e intuitiva, che permetta una facile consultazione
   \item Unire la comunità universitaria di Trento creando un gruppo unito e coeso
\end{enumerate}
%
Per ottenere ciò, Univents mette a disposizione due macro-sezioni all’interno dell’applicazione:
%
\begin{itemize}
   \item Una \textit{lista degli eventi} interni all’università, dove gli event manager possono pubblicare le iniziative organizzate dall’ateneo stesso e che gli studenti possono solo consultare
   \item Una \textit{bacheca interattiva} dove gli utenti stessi possono pubblicare iniziative private e/o eventi che si svolgono in vari luoghi della città, per rendere noto l’evento al resto della comunità studentesca
\end{itemize}
%
La bacheca interattiva sarà gestita da un gruppo di moderatori, che avranno la capacità di eliminare i post e i commenti che non rispettano le linee guida del sito, nonché di bloccare gli utenti.

L’intera applicazione viene gestita da uno o più amministratori di sistema, i quali possono, in ogni momento, concedere o revocare i privilegi di event manager o di moderatore agli utenti regolari, nonché di nominarne di nuovi.\\
Le figure che svolgono il ruolo di attori di questo sistema sono:
%
\begin{itemize}
   \item Studente
   \item Moderatore
   \item Event Manager
   \item Amministratore
\end{itemize}
%
\section*{Requisiti funzionali}
%
   \subsection{Login}
   Tutti gli utenti, indipendentemente dal loro ruolo, verranno reindirizzati al sito di login offerto dall’università di Trento perché possano eseguire l’accesso tramite le credenziali di ateneo.
   \subsection{Gestione eventi interni}
   \subsection{Gestione eventi esterni}
   \subsection{Commenti}
   \subsection{Moderazione della bacheca}
   \subsection{Salvataggio degli eventi}
   \subsection{Filtraggio degli eventi}
%
\end{document}